\documentclass{article}
\usepackage{geometry}\geometry{a4paper}
\usepackage[T1]{fontenc}
\usepackage[latin1]{inputenc}
\usepackage{ae}
\usepackage[cellrefs]{exceltex}


\begin{document}


\section{cells}


\begin{description}
\item[test1!A1 (A1):] \inccell{test1!A1}
\item[test1!C16491 (C16491):]\inccell{test1!C16491}
\item[test1!G17: (colored, strikeout, bold)] \inccell{test1!G17}
\item[test1!Z2 (Z2):] \inccell{test1!Z2}
\item[test1!AA2 (AA2):] \inccell{test1!AA2}
\item[test1!BA2 (BA2):] \inccell{test1!BA2}
\item[test1!B70 (empty):] \inccell{test1!B70}
\item[test2!B70 (no such sheet):] \inccell{test.xls!test2!B70}
\item[test2!CB (invlaid cell):] \inccell{test.xls!test1!CB}

\end{description}


\section{tables}



\subsubsection{misc formatted cells}
\begin{tabular}{lll}
  \hline
  column 1 & colum 2 & column 3\\
  \hline
  \inctab{test1!f13!h23}
  \hline
\end{tabular}

\subsubsection{latex control characters}
\begin{tabular}{lll}
  \hline
  column 1 & colum 2 \\
  \hline
  \inctab{test1!B32!D36}
  \hline
\end{tabular}


\subsubsection{Integer Table}
\begin{tabular}{lll}
  \hline
  column 1 & colum 2 & colum 3 \\
  \hline
  \inctab{test1!B4!D8}
  \hline
\end{tabular}

\subsubsection{float table}
\begin{tabular}{lll}
  \hline
  column 1 & colum 2 & colum 3 \\
  \hline
  \inctab{test1!B12!D16}
  \hline
\end{tabular}


\subsubsection{scientific numbers}
\begin{tabular}{lll}
  \hline
  column 1 & colum 2 & colum 3 \\
  \hline
  \inctab{test1!B20!D24}
  \hline
\end{tabular}

\subsubsection{empty table}
\begin{tabular}{lll}
  \hline
  column 1 & colum 2 & colum 3 \\
  \hline
  \inctab{test1!B60!D64}
  \hline
\end{tabular}



\end{document}
